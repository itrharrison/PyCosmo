% Generated by Sphinx.
\def\sphinxdocclass{report}
\documentclass[letterpaper,10pt,english]{sphinxmanual}
\usepackage[utf8]{inputenc}
\DeclareUnicodeCharacter{00A0}{\nobreakspace}
\usepackage[T1]{fontenc}
\usepackage{babel}
\usepackage{times}
\usepackage[Bjarne]{fncychap}
\usepackage{longtable}
\usepackage{sphinx}
\usepackage{multirow}


\title{PyCosmo Documentation}
\date{July 24, 2013}
\release{1.0}
\author{Ian Harrison \& Christopher Lovell}
\newcommand{\sphinxlogo}{}
\renewcommand{\releasename}{Release}
\makeindex

\makeatletter
\def\PYG@reset{\let\PYG@it=\relax \let\PYG@bf=\relax%
    \let\PYG@ul=\relax \let\PYG@tc=\relax%
    \let\PYG@bc=\relax \let\PYG@ff=\relax}
\def\PYG@tok#1{\csname PYG@tok@#1\endcsname}
\def\PYG@toks#1+{\ifx\relax#1\empty\else%
    \PYG@tok{#1}\expandafter\PYG@toks\fi}
\def\PYG@do#1{\PYG@bc{\PYG@tc{\PYG@ul{%
    \PYG@it{\PYG@bf{\PYG@ff{#1}}}}}}}
\def\PYG#1#2{\PYG@reset\PYG@toks#1+\relax+\PYG@do{#2}}

\expandafter\def\csname PYG@tok@gd\endcsname{\def\PYG@tc##1{\textcolor[rgb]{0.63,0.00,0.00}{##1}}}
\expandafter\def\csname PYG@tok@gu\endcsname{\let\PYG@bf=\textbf\def\PYG@tc##1{\textcolor[rgb]{0.50,0.00,0.50}{##1}}}
\expandafter\def\csname PYG@tok@gt\endcsname{\def\PYG@tc##1{\textcolor[rgb]{0.00,0.25,0.82}{##1}}}
\expandafter\def\csname PYG@tok@gs\endcsname{\let\PYG@bf=\textbf}
\expandafter\def\csname PYG@tok@gr\endcsname{\def\PYG@tc##1{\textcolor[rgb]{1.00,0.00,0.00}{##1}}}
\expandafter\def\csname PYG@tok@cm\endcsname{\let\PYG@it=\textit\def\PYG@tc##1{\textcolor[rgb]{0.25,0.50,0.56}{##1}}}
\expandafter\def\csname PYG@tok@vg\endcsname{\def\PYG@tc##1{\textcolor[rgb]{0.73,0.38,0.84}{##1}}}
\expandafter\def\csname PYG@tok@m\endcsname{\def\PYG@tc##1{\textcolor[rgb]{0.13,0.50,0.31}{##1}}}
\expandafter\def\csname PYG@tok@mh\endcsname{\def\PYG@tc##1{\textcolor[rgb]{0.13,0.50,0.31}{##1}}}
\expandafter\def\csname PYG@tok@cs\endcsname{\def\PYG@tc##1{\textcolor[rgb]{0.25,0.50,0.56}{##1}}\def\PYG@bc##1{\setlength{\fboxsep}{0pt}\colorbox[rgb]{1.00,0.94,0.94}{\strut ##1}}}
\expandafter\def\csname PYG@tok@ge\endcsname{\let\PYG@it=\textit}
\expandafter\def\csname PYG@tok@vc\endcsname{\def\PYG@tc##1{\textcolor[rgb]{0.73,0.38,0.84}{##1}}}
\expandafter\def\csname PYG@tok@il\endcsname{\def\PYG@tc##1{\textcolor[rgb]{0.13,0.50,0.31}{##1}}}
\expandafter\def\csname PYG@tok@go\endcsname{\def\PYG@tc##1{\textcolor[rgb]{0.19,0.19,0.19}{##1}}}
\expandafter\def\csname PYG@tok@cp\endcsname{\def\PYG@tc##1{\textcolor[rgb]{0.00,0.44,0.13}{##1}}}
\expandafter\def\csname PYG@tok@gi\endcsname{\def\PYG@tc##1{\textcolor[rgb]{0.00,0.63,0.00}{##1}}}
\expandafter\def\csname PYG@tok@gh\endcsname{\let\PYG@bf=\textbf\def\PYG@tc##1{\textcolor[rgb]{0.00,0.00,0.50}{##1}}}
\expandafter\def\csname PYG@tok@ni\endcsname{\let\PYG@bf=\textbf\def\PYG@tc##1{\textcolor[rgb]{0.84,0.33,0.22}{##1}}}
\expandafter\def\csname PYG@tok@nl\endcsname{\let\PYG@bf=\textbf\def\PYG@tc##1{\textcolor[rgb]{0.00,0.13,0.44}{##1}}}
\expandafter\def\csname PYG@tok@nn\endcsname{\let\PYG@bf=\textbf\def\PYG@tc##1{\textcolor[rgb]{0.05,0.52,0.71}{##1}}}
\expandafter\def\csname PYG@tok@no\endcsname{\def\PYG@tc##1{\textcolor[rgb]{0.38,0.68,0.84}{##1}}}
\expandafter\def\csname PYG@tok@na\endcsname{\def\PYG@tc##1{\textcolor[rgb]{0.25,0.44,0.63}{##1}}}
\expandafter\def\csname PYG@tok@nb\endcsname{\def\PYG@tc##1{\textcolor[rgb]{0.00,0.44,0.13}{##1}}}
\expandafter\def\csname PYG@tok@nc\endcsname{\let\PYG@bf=\textbf\def\PYG@tc##1{\textcolor[rgb]{0.05,0.52,0.71}{##1}}}
\expandafter\def\csname PYG@tok@nd\endcsname{\let\PYG@bf=\textbf\def\PYG@tc##1{\textcolor[rgb]{0.33,0.33,0.33}{##1}}}
\expandafter\def\csname PYG@tok@ne\endcsname{\def\PYG@tc##1{\textcolor[rgb]{0.00,0.44,0.13}{##1}}}
\expandafter\def\csname PYG@tok@nf\endcsname{\def\PYG@tc##1{\textcolor[rgb]{0.02,0.16,0.49}{##1}}}
\expandafter\def\csname PYG@tok@si\endcsname{\let\PYG@it=\textit\def\PYG@tc##1{\textcolor[rgb]{0.44,0.63,0.82}{##1}}}
\expandafter\def\csname PYG@tok@s2\endcsname{\def\PYG@tc##1{\textcolor[rgb]{0.25,0.44,0.63}{##1}}}
\expandafter\def\csname PYG@tok@vi\endcsname{\def\PYG@tc##1{\textcolor[rgb]{0.73,0.38,0.84}{##1}}}
\expandafter\def\csname PYG@tok@nt\endcsname{\let\PYG@bf=\textbf\def\PYG@tc##1{\textcolor[rgb]{0.02,0.16,0.45}{##1}}}
\expandafter\def\csname PYG@tok@nv\endcsname{\def\PYG@tc##1{\textcolor[rgb]{0.73,0.38,0.84}{##1}}}
\expandafter\def\csname PYG@tok@s1\endcsname{\def\PYG@tc##1{\textcolor[rgb]{0.25,0.44,0.63}{##1}}}
\expandafter\def\csname PYG@tok@gp\endcsname{\let\PYG@bf=\textbf\def\PYG@tc##1{\textcolor[rgb]{0.78,0.36,0.04}{##1}}}
\expandafter\def\csname PYG@tok@sh\endcsname{\def\PYG@tc##1{\textcolor[rgb]{0.25,0.44,0.63}{##1}}}
\expandafter\def\csname PYG@tok@ow\endcsname{\let\PYG@bf=\textbf\def\PYG@tc##1{\textcolor[rgb]{0.00,0.44,0.13}{##1}}}
\expandafter\def\csname PYG@tok@sx\endcsname{\def\PYG@tc##1{\textcolor[rgb]{0.78,0.36,0.04}{##1}}}
\expandafter\def\csname PYG@tok@bp\endcsname{\def\PYG@tc##1{\textcolor[rgb]{0.00,0.44,0.13}{##1}}}
\expandafter\def\csname PYG@tok@c1\endcsname{\let\PYG@it=\textit\def\PYG@tc##1{\textcolor[rgb]{0.25,0.50,0.56}{##1}}}
\expandafter\def\csname PYG@tok@kc\endcsname{\let\PYG@bf=\textbf\def\PYG@tc##1{\textcolor[rgb]{0.00,0.44,0.13}{##1}}}
\expandafter\def\csname PYG@tok@c\endcsname{\let\PYG@it=\textit\def\PYG@tc##1{\textcolor[rgb]{0.25,0.50,0.56}{##1}}}
\expandafter\def\csname PYG@tok@mf\endcsname{\def\PYG@tc##1{\textcolor[rgb]{0.13,0.50,0.31}{##1}}}
\expandafter\def\csname PYG@tok@err\endcsname{\def\PYG@bc##1{\setlength{\fboxsep}{0pt}\fcolorbox[rgb]{1.00,0.00,0.00}{1,1,1}{\strut ##1}}}
\expandafter\def\csname PYG@tok@kd\endcsname{\let\PYG@bf=\textbf\def\PYG@tc##1{\textcolor[rgb]{0.00,0.44,0.13}{##1}}}
\expandafter\def\csname PYG@tok@ss\endcsname{\def\PYG@tc##1{\textcolor[rgb]{0.32,0.47,0.09}{##1}}}
\expandafter\def\csname PYG@tok@sr\endcsname{\def\PYG@tc##1{\textcolor[rgb]{0.14,0.33,0.53}{##1}}}
\expandafter\def\csname PYG@tok@mo\endcsname{\def\PYG@tc##1{\textcolor[rgb]{0.13,0.50,0.31}{##1}}}
\expandafter\def\csname PYG@tok@mi\endcsname{\def\PYG@tc##1{\textcolor[rgb]{0.13,0.50,0.31}{##1}}}
\expandafter\def\csname PYG@tok@kn\endcsname{\let\PYG@bf=\textbf\def\PYG@tc##1{\textcolor[rgb]{0.00,0.44,0.13}{##1}}}
\expandafter\def\csname PYG@tok@o\endcsname{\def\PYG@tc##1{\textcolor[rgb]{0.40,0.40,0.40}{##1}}}
\expandafter\def\csname PYG@tok@kr\endcsname{\let\PYG@bf=\textbf\def\PYG@tc##1{\textcolor[rgb]{0.00,0.44,0.13}{##1}}}
\expandafter\def\csname PYG@tok@s\endcsname{\def\PYG@tc##1{\textcolor[rgb]{0.25,0.44,0.63}{##1}}}
\expandafter\def\csname PYG@tok@kp\endcsname{\def\PYG@tc##1{\textcolor[rgb]{0.00,0.44,0.13}{##1}}}
\expandafter\def\csname PYG@tok@w\endcsname{\def\PYG@tc##1{\textcolor[rgb]{0.73,0.73,0.73}{##1}}}
\expandafter\def\csname PYG@tok@kt\endcsname{\def\PYG@tc##1{\textcolor[rgb]{0.56,0.13,0.00}{##1}}}
\expandafter\def\csname PYG@tok@sc\endcsname{\def\PYG@tc##1{\textcolor[rgb]{0.25,0.44,0.63}{##1}}}
\expandafter\def\csname PYG@tok@sb\endcsname{\def\PYG@tc##1{\textcolor[rgb]{0.25,0.44,0.63}{##1}}}
\expandafter\def\csname PYG@tok@k\endcsname{\let\PYG@bf=\textbf\def\PYG@tc##1{\textcolor[rgb]{0.00,0.44,0.13}{##1}}}
\expandafter\def\csname PYG@tok@se\endcsname{\let\PYG@bf=\textbf\def\PYG@tc##1{\textcolor[rgb]{0.25,0.44,0.63}{##1}}}
\expandafter\def\csname PYG@tok@sd\endcsname{\let\PYG@it=\textit\def\PYG@tc##1{\textcolor[rgb]{0.25,0.44,0.63}{##1}}}

\def\PYGZbs{\char`\\}
\def\PYGZus{\char`\_}
\def\PYGZob{\char`\{}
\def\PYGZcb{\char`\}}
\def\PYGZca{\char`\^}
\def\PYGZam{\char`\&}
\def\PYGZlt{\char`\<}
\def\PYGZgt{\char`\>}
\def\PYGZsh{\char`\#}
\def\PYGZpc{\char`\%}
\def\PYGZdl{\char`\$}
\def\PYGZti{\char`\~}
% for compatibility with earlier versions
\def\PYGZat{@}
\def\PYGZlb{[}
\def\PYGZrb{]}
\makeatother

\begin{document}

\maketitle
\tableofcontents
\phantomsection\label{index::doc}


Contents:


\chapter{PyCosmo Package}
\label{PyCosmo::doc}\label{PyCosmo:pycosmo-package}\label{PyCosmo:welcome-to-pycosmo-s-documentation}

\section{\texttt{cluster} Module}
\label{PyCosmo:cluster-module}\label{PyCosmo:module-PyCosmo.cluster}\index{PyCosmo.cluster (module)}
\code{cmb} Module
\index{clust (class in PyCosmo.cluster)}

\begin{fulllineitems}
\phantomsection\label{PyCosmo:PyCosmo.cluster.clust}\pysiglinewithargsret{\strong{class }\code{PyCosmo.cluster.}\bfcode{clust}}{\emph{lnm}, \emph{redshift}, \emph{cosm}}{}~\index{display() (PyCosmo.cluster.clust method)}

\begin{fulllineitems}
\phantomsection\label{PyCosmo:PyCosmo.cluster.clust.display}\pysiglinewithargsret{\bfcode{display}}{\emph{pixsize=None}}{}
\end{fulllineitems}

\index{info() (PyCosmo.cluster.clust method)}

\begin{fulllineitems}
\phantomsection\label{PyCosmo:PyCosmo.cluster.clust.info}\pysiglinewithargsret{\bfcode{info}}{}{}
\end{fulllineitems}

\index{make\_ymap() (PyCosmo.cluster.clust method)}

\begin{fulllineitems}
\phantomsection\label{PyCosmo:PyCosmo.cluster.clust.make_ymap}\pysiglinewithargsret{\bfcode{make\_ymap}}{\emph{pixsize=None}}{}
\end{fulllineitems}

\index{trim() (PyCosmo.cluster.clust method)}

\begin{fulllineitems}
\phantomsection\label{PyCosmo:PyCosmo.cluster.clust.trim}\pysiglinewithargsret{\bfcode{trim}}{\emph{edge}, \emph{npix}}{}
\end{fulllineitems}

\index{y() (PyCosmo.cluster.clust method)}

\begin{fulllineitems}
\phantomsection\label{PyCosmo:PyCosmo.cluster.clust.y}\pysiglinewithargsret{\bfcode{y}}{\emph{theta}}{}
\end{fulllineitems}


\end{fulllineitems}



\bigskip\hrule{}\bigskip

\phantomsection\label{PyCosmo:module-PyCosmo.cmb}\index{PyCosmo.cmb (module)}
Python implementation of Geraint Pratten's code
for generating flat-sky CMB maps...
\index{CMBFlatMap (class in PyCosmo.cmb)}

\begin{fulllineitems}
\phantomsection\label{PyCosmo:PyCosmo.cmb.CMBFlatMap}\pysiglinewithargsret{\strong{class }\code{PyCosmo.cmb.}\bfcode{CMBFlatMap}}{\emph{mapsize=10.0}, \emph{pixels=1024}, \emph{cosm=\textless{}PyCosmo.cosmology.Cosmology instance at 0x37e44d0\textgreater{}}}{}~\index{add\_cluster\_ymap() (PyCosmo.cmb.CMBFlatMap method)}

\begin{fulllineitems}
\phantomsection\label{PyCosmo:PyCosmo.cmb.CMBFlatMap.add_cluster_ymap}\pysiglinewithargsret{\bfcode{add\_cluster\_ymap}}{\emph{cl}}{}
Add Compton Y from an object of the Cluster class to the map as temp.
TO FIX: edges.

\end{fulllineitems}

\index{add\_noise() (PyCosmo.cmb.CMBFlatMap method)}

\begin{fulllineitems}
\phantomsection\label{PyCosmo:PyCosmo.cmb.CMBFlatMap.add_noise}\pysiglinewithargsret{\bfcode{add\_noise}}{\emph{temp}}{}
Add per-pixel Gaussian random noise.

\end{fulllineitems}

\index{add\_ymap() (PyCosmo.cmb.CMBFlatMap method)}

\begin{fulllineitems}
\phantomsection\label{PyCosmo:PyCosmo.cmb.CMBFlatMap.add_ymap}\pysiglinewithargsret{\bfcode{add\_ymap}}{\emph{ymap}, \emph{freq=1.0}}{}
Add a Compton Y map (e.g. from LSS tSZ) to the map as temperature.
DeltaT(x) = -2*T\_\{CMB\}*Y(x)

\end{fulllineitems}

\index{apply\_beam() (PyCosmo.cmb.CMBFlatMap method)}

\begin{fulllineitems}
\phantomsection\label{PyCosmo:PyCosmo.cmb.CMBFlatMap.apply_beam}\pysiglinewithargsret{\bfcode{apply\_beam}}{\emph{beamtype}, \emph{fwhm}}{}
Apply a beam function to map.
Currently `beamtype' is redundant and a Gaussian beam is always used.

\textbf{**fwhm is in degrees**}

\end{fulllineitems}

\index{beta\_profile\_map() (PyCosmo.cmb.CMBFlatMap method)}

\begin{fulllineitems}
\phantomsection\label{PyCosmo:PyCosmo.cmb.CMBFlatMap.beta_profile_map}\pysiglinewithargsret{\bfcode{beta\_profile\_map}}{\emph{theta\_crit}, \emph{cutoff\_factor=10.0}, \emph{deltaT0=1.0}}{}
\end{fulllineitems}

\index{build\_Pk() (PyCosmo.cmb.CMBFlatMap method)}

\begin{fulllineitems}
\phantomsection\label{PyCosmo:PyCosmo.cmb.CMBFlatMap.build_Pk}\pysiglinewithargsret{\bfcode{build\_Pk}}{\emph{cosm}}{}
Calculate correct flat-sky P(k) from external Cls (angular power spectrum).

\end{fulllineitems}

\index{display() (PyCosmo.cmb.CMBFlatMap method)}

\begin{fulllineitems}
\phantomsection\label{PyCosmo:PyCosmo.cmb.CMBFlatMap.display}\pysiglinewithargsret{\bfcode{display}}{}{}
Plot the flat-sky CMB.

\end{fulllineitems}

\index{matched\_filter() (PyCosmo.cmb.CMBFlatMap method)}

\begin{fulllineitems}
\phantomsection\label{PyCosmo:PyCosmo.cmb.CMBFlatMap.matched_filter}\pysiglinewithargsret{\bfcode{matched\_filter}}{\emph{noisesigma}}{}
\end{fulllineitems}

\index{matched\_filter\_Melin() (PyCosmo.cmb.CMBFlatMap method)}

\begin{fulllineitems}
\phantomsection\label{PyCosmo:PyCosmo.cmb.CMBFlatMap.matched_filter_Melin}\pysiglinewithargsret{\bfcode{matched\_filter\_Melin}}{\emph{theta\_c}, \emph{xfilt}, \emph{noisesigma}, \emph{deltaT0=1.0}}{}
Matched filter with a spherical beta profile as in SPT paper

\end{fulllineitems}


\end{fulllineitems}



\section{\texttt{constants} Module}
\label{PyCosmo:module-PyCosmo.constants}\label{PyCosmo:constants-module}\index{PyCosmo.constants (module)}
constants (dictionary)


\section{\texttt{cosmology} Module}
\label{PyCosmo:module-PyCosmo.cosmology}\label{PyCosmo:cosmology-module}\index{PyCosmo.cosmology (module)}
(MILDLY) USEFUL COSMOLOGY CLASS.
NOTE ALL UNITS ARE WRT h\textasciicircum{}-1

Much of the cosmography is from (as ever) Hogg arXiv:astro-ph/9905116

(C) IAN HARRISON
2012-
\href{mailto:IAN.HARRISON@ASTRO.CF.AC.UK}{IAN.HARRISON@ASTRO.CF.AC.UK}
\index{Cosmology (class in PyCosmo.cosmology)}

\begin{fulllineitems}
\phantomsection\label{PyCosmo:PyCosmo.cosmology.Cosmology}\pysiglinewithargsret{\strong{class }\code{PyCosmo.cosmology.}\bfcode{Cosmology}}{\emph{dc=1.686}, \emph{h0=0.702}, \emph{om=0.274}, \emph{ode=0.725}, \emph{w0=-1.0}, \emph{ob=0.0458}, \emph{o\_r=8.6e-05}, \emph{o\_k=0.0}, \emph{f\_nl=0.0}, \emph{tau\_r=0.087}, \emph{z\_r=10.4}, \emph{ns=0.96}, \emph{hmf\_type='tinker'}}{}~\index{D\_a() (PyCosmo.cosmology.Cosmology method)}

\begin{fulllineitems}
\phantomsection\label{PyCosmo:PyCosmo.cosmology.Cosmology.D_a}\pysiglinewithargsret{\bfcode{D\_a}}{\emph{z}}{}
Angular diameter distance to redshift z.

\end{fulllineitems}

\index{D\_c() (PyCosmo.cosmology.Cosmology method)}

\begin{fulllineitems}
\phantomsection\label{PyCosmo:PyCosmo.cosmology.Cosmology.D_c}\pysiglinewithargsret{\bfcode{D\_c}}{\emph{z}}{}
Comoving radial distance to redshift z.

\end{fulllineitems}

\index{D\_l() (PyCosmo.cosmology.Cosmology method)}

\begin{fulllineitems}
\phantomsection\label{PyCosmo:PyCosmo.cosmology.Cosmology.D_l}\pysiglinewithargsret{\bfcode{D\_l}}{\emph{z}}{}
Luminosity distance to redshift z.

\end{fulllineitems}

\index{H() (PyCosmo.cosmology.Cosmology method)}

\begin{fulllineitems}
\phantomsection\label{PyCosmo:PyCosmo.cosmology.Cosmology.H}\pysiglinewithargsret{\bfcode{H}}{\emph{z}}{}
Hubble function \textbf{*upper*} H(z).

\end{fulllineitems}

\index{O\_m() (PyCosmo.cosmology.Cosmology method)}

\begin{fulllineitems}
\phantomsection\label{PyCosmo:PyCosmo.cosmology.Cosmology.O_m}\pysiglinewithargsret{\bfcode{O\_m}}{\emph{z}}{}
Omega matter.

\end{fulllineitems}

\index{V\_between() (PyCosmo.cosmology.Cosmology method)}

\begin{fulllineitems}
\phantomsection\label{PyCosmo:PyCosmo.cosmology.Cosmology.V_between}\pysiglinewithargsret{\bfcode{V\_between}}{\emph{z\_min}, \emph{z\_max}}{}
Volume between two redshifts.

\end{fulllineitems}

\index{computeLittleNinZBin() (PyCosmo.cosmology.Cosmology method)}

\begin{fulllineitems}
\phantomsection\label{PyCosmo:PyCosmo.cosmology.Cosmology.computeLittleNinZBin}\pysiglinewithargsret{\bfcode{computeLittleNinZBin}}{\emph{lnm\_min}, \emph{lnm\_max}, \emph{z}}{}
Total number of haloes within a given mass bin at fixed redshift.

\end{fulllineitems}

\index{computeNinBin() (PyCosmo.cosmology.Cosmology method)}

\begin{fulllineitems}
\phantomsection\label{PyCosmo:PyCosmo.cosmology.Cosmology.computeNinBin}\pysiglinewithargsret{\bfcode{computeNinBin}}{\emph{z\_min}, \emph{z\_max}, \emph{lnm\_min}, \emph{lnm\_max}}{}
Total number of dark matter haloes expected within a given mass,
redshift bin.

\end{fulllineitems}

\index{dNdlnm0dz() (PyCosmo.cosmology.Cosmology method)}

\begin{fulllineitems}
\phantomsection\label{PyCosmo:PyCosmo.cosmology.Cosmology.dNdlnm0dz}\pysiglinewithargsret{\bfcode{dNdlnm0dz}}{\emph{lnm}, \emph{z}}{}
Total number of dark matter haloes,
of equivalent mass at redshift zero, at a given redshift.
Product of dndlnm * dVdz

\end{fulllineitems}

\index{dNdlnmdz() (PyCosmo.cosmology.Cosmology method)}

\begin{fulllineitems}
\phantomsection\label{PyCosmo:PyCosmo.cosmology.Cosmology.dNdlnmdz}\pysiglinewithargsret{\bfcode{dNdlnmdz}}{\emph{lnm}, \emph{z}}{}
Total number of dark matter haloes at a given redshift.
Product of dndlnm * dVdz

\end{fulllineitems}

\index{display() (PyCosmo.cosmology.Cosmology method)}

\begin{fulllineitems}
\phantomsection\label{PyCosmo:PyCosmo.cosmology.Cosmology.display}\pysiglinewithargsret{\bfcode{display}}{}{}
Displays what you're working with.

\end{fulllineitems}

\index{dist\_mod() (PyCosmo.cosmology.Cosmology method)}

\begin{fulllineitems}
\phantomsection\label{PyCosmo:PyCosmo.cosmology.Cosmology.dist_mod}\pysiglinewithargsret{\bfcode{dist\_mod}}{\emph{z}}{}
Distance modulus.
5 * log(D\_l(z)) + 25

\end{fulllineitems}

\index{dndlnm() (PyCosmo.cosmology.Cosmology method)}

\begin{fulllineitems}
\phantomsection\label{PyCosmo:PyCosmo.cosmology.Cosmology.dndlnm}\pysiglinewithargsret{\bfcode{dndlnm}}{\emph{lnm}, \emph{z}}{}
Comoving number density of dark matter haloes in logarithmic m.

\end{fulllineitems}

\index{dvdz() (PyCosmo.cosmology.Cosmology method)}

\begin{fulllineitems}
\phantomsection\label{PyCosmo:PyCosmo.cosmology.Cosmology.dvdz}\pysiglinewithargsret{\bfcode{dvdz}}{\emph{z}}{}
Comoving volume element at redshift z.

\end{fulllineitems}

\index{eta() (PyCosmo.cosmology.Cosmology method)}

\begin{fulllineitems}
\phantomsection\label{PyCosmo:PyCosmo.cosmology.Cosmology.eta}\pysiglinewithargsret{\bfcode{eta}}{\emph{z}}{}
Size of particle horizon at redshift z

\end{fulllineitems}

\index{growth() (PyCosmo.cosmology.Cosmology method)}

\begin{fulllineitems}
\phantomsection\label{PyCosmo:PyCosmo.cosmology.Cosmology.growth}\pysiglinewithargsret{\bfcode{growth}}{\emph{z}}{}
Linear growth function.

\end{fulllineitems}

\index{h() (PyCosmo.cosmology.Cosmology method)}

\begin{fulllineitems}
\phantomsection\label{PyCosmo:PyCosmo.cosmology.Cosmology.h}\pysiglinewithargsret{\bfcode{h}}{\emph{z}}{}
Hubble function \textbf{*little*} h(z).

\end{fulllineitems}

\index{rho\_c() (PyCosmo.cosmology.Cosmology method)}

\begin{fulllineitems}
\phantomsection\label{PyCosmo:PyCosmo.cosmology.Cosmology.rho_c}\pysiglinewithargsret{\bfcode{rho\_c}}{}{}
Critical density

\end{fulllineitems}

\index{rho\_m() (PyCosmo.cosmology.Cosmology method)}

\begin{fulllineitems}
\phantomsection\label{PyCosmo:PyCosmo.cosmology.Cosmology.rho_m}\pysiglinewithargsret{\bfcode{rho\_m}}{\emph{z}}{}
Average density of Universe at redshift z

\end{fulllineitems}

\index{set\_hmf() (PyCosmo.cosmology.Cosmology method)}

\begin{fulllineitems}
\phantomsection\label{PyCosmo:PyCosmo.cosmology.Cosmology.set_hmf}\pysiglinewithargsret{\bfcode{set\_hmf}}{\emph{set\_mf}}{}
Set method for the HMF within the Cosmology

\end{fulllineitems}

\index{set\_powspec() (PyCosmo.cosmology.Cosmology method)}

\begin{fulllineitems}
\phantomsection\label{PyCosmo:PyCosmo.cosmology.Cosmology.set_powspec}\pysiglinewithargsret{\bfcode{set\_powspec}}{\emph{set\_pk}}{}
Set method for power spectrum within the Cosmology

\end{fulllineitems}

\index{t\_lookback() (PyCosmo.cosmology.Cosmology method)}

\begin{fulllineitems}
\phantomsection\label{PyCosmo:PyCosmo.cosmology.Cosmology.t_lookback}\pysiglinewithargsret{\bfcode{t\_lookback}}{\emph{z}}{}
Lookback time to redshift z

\end{fulllineitems}


\end{fulllineitems}



\section{\texttt{cosmology\_prevec} Module}
\label{PyCosmo:module-PyCosmo.cosmology_prevec}\label{PyCosmo:cosmology-prevec-module}\index{PyCosmo.cosmology\_prevec (module)}
(MILDLY) USEFUL COSMOLOGY CLASS.
NOTE ALL UNITS ARE WRT h\textasciicircum{}-1

Much of the cosmography is from (as ever) Hogg arXiv:astro-ph/9905116

(C) IAN HARRISON
2012-
\href{mailto:IAN.HARRISON@ASTRO.CF.AC.UK}{IAN.HARRISON@ASTRO.CF.AC.UK}
\index{Cosmology (class in PyCosmo.cosmology\_prevec)}

\begin{fulllineitems}
\phantomsection\label{PyCosmo:PyCosmo.cosmology_prevec.Cosmology}\pysiglinewithargsret{\strong{class }\code{PyCosmo.cosmology\_prevec.}\bfcode{Cosmology}}{\emph{dc=1.686}, \emph{h0=0.702}, \emph{om=0.274}, \emph{ode=0.725}, \emph{w0=-1.0}, \emph{ob=0.0458}, \emph{o\_r=8.6e-05}, \emph{o\_k=0.0}, \emph{f\_nl=0.0}, \emph{tau\_r=0.087}, \emph{z\_r=10.4}, \emph{ns=0.96}}{}~\index{D\_a() (PyCosmo.cosmology\_prevec.Cosmology method)}

\begin{fulllineitems}
\phantomsection\label{PyCosmo:PyCosmo.cosmology_prevec.Cosmology.D_a}\pysiglinewithargsret{\bfcode{D\_a}}{\emph{z}}{}
Angular diameter distance to redshift z.

\end{fulllineitems}

\index{D\_c() (PyCosmo.cosmology\_prevec.Cosmology method)}

\begin{fulllineitems}
\phantomsection\label{PyCosmo:PyCosmo.cosmology_prevec.Cosmology.D_c}\pysiglinewithargsret{\bfcode{D\_c}}{\emph{z}}{}
Comoving radial distance to redshift z.

\end{fulllineitems}

\index{D\_l() (PyCosmo.cosmology\_prevec.Cosmology method)}

\begin{fulllineitems}
\phantomsection\label{PyCosmo:PyCosmo.cosmology_prevec.Cosmology.D_l}\pysiglinewithargsret{\bfcode{D\_l}}{\emph{z}}{}
Luminosity distance to redshift z.

\end{fulllineitems}

\index{H() (PyCosmo.cosmology\_prevec.Cosmology method)}

\begin{fulllineitems}
\phantomsection\label{PyCosmo:PyCosmo.cosmology_prevec.Cosmology.H}\pysiglinewithargsret{\bfcode{H}}{\emph{z}}{}
Hubble function \textbf{*upper*} H(z).

\end{fulllineitems}

\index{O\_m() (PyCosmo.cosmology\_prevec.Cosmology method)}

\begin{fulllineitems}
\phantomsection\label{PyCosmo:PyCosmo.cosmology_prevec.Cosmology.O_m}\pysiglinewithargsret{\bfcode{O\_m}}{\emph{z}}{}
Omega matter.

\end{fulllineitems}

\index{V\_between() (PyCosmo.cosmology\_prevec.Cosmology method)}

\begin{fulllineitems}
\phantomsection\label{PyCosmo:PyCosmo.cosmology_prevec.Cosmology.V_between}\pysiglinewithargsret{\bfcode{V\_between}}{\emph{z\_min}, \emph{z\_max}}{}
Volume between two redshifts.

\end{fulllineitems}

\index{computeNinBin() (PyCosmo.cosmology\_prevec.Cosmology method)}

\begin{fulllineitems}
\phantomsection\label{PyCosmo:PyCosmo.cosmology_prevec.Cosmology.computeNinBin}\pysiglinewithargsret{\bfcode{computeNinBin}}{\emph{z\_min}, \emph{z\_max}, \emph{lnm\_min}, \emph{lnm\_max}}{}
Total number of dark matter haloes expected within a given mass,
redshift bin.

\end{fulllineitems}

\index{dNdlnm0dz() (PyCosmo.cosmology\_prevec.Cosmology method)}

\begin{fulllineitems}
\phantomsection\label{PyCosmo:PyCosmo.cosmology_prevec.Cosmology.dNdlnm0dz}\pysiglinewithargsret{\bfcode{dNdlnm0dz}}{\emph{lnm}, \emph{z}}{}
Total number of dark matter haloes,
of equivalent mass at redshift zero, at a given redshift.
Product of dndlnm * dVdz

\end{fulllineitems}

\index{dNdlnmdz() (PyCosmo.cosmology\_prevec.Cosmology method)}

\begin{fulllineitems}
\phantomsection\label{PyCosmo:PyCosmo.cosmology_prevec.Cosmology.dNdlnmdz}\pysiglinewithargsret{\bfcode{dNdlnmdz}}{\emph{lnm}, \emph{z}}{}
Total number of dark matter haloes at a given redshift.
Product of dndlnm * dVdz

\end{fulllineitems}

\index{display() (PyCosmo.cosmology\_prevec.Cosmology method)}

\begin{fulllineitems}
\phantomsection\label{PyCosmo:PyCosmo.cosmology_prevec.Cosmology.display}\pysiglinewithargsret{\bfcode{display}}{}{}
Displays what you're working with.

\end{fulllineitems}

\index{dist\_mod() (PyCosmo.cosmology\_prevec.Cosmology method)}

\begin{fulllineitems}
\phantomsection\label{PyCosmo:PyCosmo.cosmology_prevec.Cosmology.dist_mod}\pysiglinewithargsret{\bfcode{dist\_mod}}{\emph{z}}{}
Distance modulus.
5 * log(D\_l(z)) + 25

\end{fulllineitems}

\index{dndlnm() (PyCosmo.cosmology\_prevec.Cosmology method)}

\begin{fulllineitems}
\phantomsection\label{PyCosmo:PyCosmo.cosmology_prevec.Cosmology.dndlnm}\pysiglinewithargsret{\bfcode{dndlnm}}{\emph{lnm}, \emph{z}}{}
Comoving number density of dark matter haloes in logarithmic m.

\end{fulllineitems}

\index{dvdz() (PyCosmo.cosmology\_prevec.Cosmology method)}

\begin{fulllineitems}
\phantomsection\label{PyCosmo:PyCosmo.cosmology_prevec.Cosmology.dvdz}\pysiglinewithargsret{\bfcode{dvdz}}{\emph{z}}{}
Comoving volume element at redshift z.

\end{fulllineitems}

\index{eta() (PyCosmo.cosmology\_prevec.Cosmology method)}

\begin{fulllineitems}
\phantomsection\label{PyCosmo:PyCosmo.cosmology_prevec.Cosmology.eta}\pysiglinewithargsret{\bfcode{eta}}{\emph{z}}{}
Size of particle horizon at redshift z

\end{fulllineitems}

\index{growth() (PyCosmo.cosmology\_prevec.Cosmology method)}

\begin{fulllineitems}
\phantomsection\label{PyCosmo:PyCosmo.cosmology_prevec.Cosmology.growth}\pysiglinewithargsret{\bfcode{growth}}{\emph{z}}{}
Linear growth function.

\end{fulllineitems}

\index{h() (PyCosmo.cosmology\_prevec.Cosmology method)}

\begin{fulllineitems}
\phantomsection\label{PyCosmo:PyCosmo.cosmology_prevec.Cosmology.h}\pysiglinewithargsret{\bfcode{h}}{\emph{z}}{}
Hubble function \textbf{*little*} h(z).

\end{fulllineitems}

\index{rho\_c() (PyCosmo.cosmology\_prevec.Cosmology method)}

\begin{fulllineitems}
\phantomsection\label{PyCosmo:PyCosmo.cosmology_prevec.Cosmology.rho_c}\pysiglinewithargsret{\bfcode{rho\_c}}{}{}
Critical density

\end{fulllineitems}

\index{rho\_m() (PyCosmo.cosmology\_prevec.Cosmology method)}

\begin{fulllineitems}
\phantomsection\label{PyCosmo:PyCosmo.cosmology_prevec.Cosmology.rho_m}\pysiglinewithargsret{\bfcode{rho\_m}}{\emph{z}}{}
Average density of Universe at redshift z

\end{fulllineitems}

\index{set\_hmf() (PyCosmo.cosmology\_prevec.Cosmology method)}

\begin{fulllineitems}
\phantomsection\label{PyCosmo:PyCosmo.cosmology_prevec.Cosmology.set_hmf}\pysiglinewithargsret{\bfcode{set\_hmf}}{\emph{set\_mf}}{}
Set method for the HMF within the Cosmology

\end{fulllineitems}

\index{set\_powspec() (PyCosmo.cosmology\_prevec.Cosmology method)}

\begin{fulllineitems}
\phantomsection\label{PyCosmo:PyCosmo.cosmology_prevec.Cosmology.set_powspec}\pysiglinewithargsret{\bfcode{set\_powspec}}{\emph{set\_pk}}{}
Set method for power spectrum within the Cosmology

\end{fulllineitems}

\index{t\_lookback() (PyCosmo.cosmology\_prevec.Cosmology method)}

\begin{fulllineitems}
\phantomsection\label{PyCosmo:PyCosmo.cosmology_prevec.Cosmology.t_lookback}\pysiglinewithargsret{\bfcode{t\_lookback}}{\emph{z}}{}
Lookback time to redshift z

\end{fulllineitems}


\end{fulllineitems}



\section{\texttt{hmf} Module}
\label{PyCosmo:hmf-module}\label{PyCosmo:module-PyCosmo.hmf}\index{PyCosmo.hmf (module)}
(HOPEFULLY) USEFUL HALO MASS FUNCTION CLASS.
NOTE ALL UNITS ARE WRT h\textasciicircum{}-1

(C) IAN HARRISON
2012-
\href{mailto:IAN.HARRISON@ASTRO.CF.AC.UK}{IAN.HARRISON@ASTRO.CF.AC.UK}
\index{Hmf (class in PyCosmo.hmf)}

\begin{fulllineitems}
\phantomsection\label{PyCosmo:PyCosmo.hmf.Hmf}\pysiglinewithargsret{\strong{class }\code{PyCosmo.hmf.}\bfcode{Hmf}}{\emph{mf\_type='tinker'}, \emph{rng\_type='pgh'}}{}~\index{display() (PyCosmo.hmf.Hmf method)}

\begin{fulllineitems}
\phantomsection\label{PyCosmo:PyCosmo.hmf.Hmf.display}\pysiglinewithargsret{\bfcode{display}}{}{}
Display method shows the name and parameters of the
current mass function

\end{fulllineitems}

\index{ps() (PyCosmo.hmf.Hmf method)}

\begin{fulllineitems}
\phantomsection\label{PyCosmo:PyCosmo.hmf.Hmf.ps}\pysiglinewithargsret{\bfcode{ps}}{\emph{sigma}, \emph{z}}{}
Press-Schechter halo mass function.
from Press, W. H., Schechter, P., 1974, ApJ, 187, 425

sigma : the mass variance on a particular scale
z : redshift

f\_ps : the collapse fraction

\end{fulllineitems}

\index{r\_pgh() (PyCosmo.hmf.Hmf method)}

\begin{fulllineitems}
\phantomsection\label{PyCosmo:PyCosmo.hmf.Hmf.r_pgh}\pysiglinewithargsret{\bfcode{r\_pgh}}{\emph{sigma}, \emph{delta\_c}, \emph{f\_nl}}{}
Paranjape-Gordon-Hotchkiss non-Gaussian correction factor for
halo mass functions.

\end{fulllineitems}

\index{st() (PyCosmo.hmf.Hmf method)}

\begin{fulllineitems}
\phantomsection\label{PyCosmo:PyCosmo.hmf.Hmf.st}\pysiglinewithargsret{\bfcode{st}}{\emph{sigma}, \emph{z}}{}
Sheth-Tormen halo massfunction, with corrections for
ellipsoidal collapse.

Equation 10 from arXiv:astro-ph/9901122

sigma : the mass variance on a particular scale
z : redshift

f\_st : the collapse fraction

\end{fulllineitems}

\index{tinker() (PyCosmo.hmf.Hmf method)}

\begin{fulllineitems}
\phantomsection\label{PyCosmo:PyCosmo.hmf.Hmf.tinker}\pysiglinewithargsret{\bfcode{tinker}}{\emph{sigma}, \emph{z}}{}
Tinker halo mass function with evolving parameters.
Equations 3, 5-8 from arXiv:0803.2706

sigma : the mass variance on a particular scale
z : redshift

f\_t : the collapse fraction

\end{fulllineitems}


\end{fulllineitems}



\section{\texttt{hmf\_extremes} Module}
\label{PyCosmo:module-PyCosmo.hmf_extremes}\label{PyCosmo:hmf-extremes-module}\index{PyCosmo.hmf\_extremes (module)}
Code for reproducing EVS of the z=0 halo mass function
(as in Harrison \& Coles 2011). New python code to supercede the old (finicky)
C++ one.
\index{evs\_bin\_pdf() (in module PyCosmo.hmf\_extremes)}

\begin{fulllineitems}
\phantomsection\label{PyCosmo:PyCosmo.hmf_extremes.evs_bin_pdf}\pysiglinewithargsret{\code{PyCosmo.hmf\_extremes.}\bfcode{evs\_bin\_pdf}}{\emph{z\_min=0.0}, \emph{z\_max=1.0}, \emph{z\_steps=200}, \emph{lnm\_min=32.236191301916641}, \emph{lnm\_max=39.143946580898778}, \emph{lnm\_steps=200}, \emph{cosm=\textless{}PyCosmo.cosmology.Cosmology instance at 0x3af5200\textgreater{}}, \emph{fsky=1.0}}{}
Calculate extreme value statistics of cold dark matter haloes in a
given mass and redshift bin.

Uses the method described in Harrison \& Coles 2012 arXiv:1111.1184

phi\_max : The EVS pdf
lnm\_arr : The x-points for the pdf

\end{fulllineitems}

\index{evs\_hypersurface\_pdf() (in module PyCosmo.hmf\_extremes)}

\begin{fulllineitems}
\phantomsection\label{PyCosmo:PyCosmo.hmf_extremes.evs_hypersurface_pdf}\pysiglinewithargsret{\code{PyCosmo.hmf\_extremes.}\bfcode{evs\_hypersurface\_pdf}}{\emph{r\_box=20.0}, \emph{lnm\_min=27.631021115928547}, \emph{lnm\_max=41.446531673892821}, \emph{redshift=0.0}, \emph{cosm=\textless{}PyCosmo.cosmology.Cosmology instance at 0x3557c20\textgreater{}}, \emph{lnm\_steps=200}}{}
Calculate Extreme Value Statistics for dark matter haloes in a given
cosmology on a specified spatial hypersurface (constant z box).

Uses the method described in Harrison \& Coles 2011 arXiv:0000.0000

phi\_max : The EVS pdf
lnm\_arr : The x-points for the pdf

\end{fulllineitems}

\index{evs\_survey() (in module PyCosmo.hmf\_extremes)}

\begin{fulllineitems}
\phantomsection\label{PyCosmo:PyCosmo.hmf_extremes.evs_survey}\pysiglinewithargsret{\code{PyCosmo.hmf\_extremes.}\bfcode{evs\_survey}}{\emph{surv=\textless{}PyCosmo.survey.Survey instance at 0x3af52d8\textgreater{}}, \emph{cosm=\textless{}PyCosmo.cosmology.Cosmology instance at 0x3af5320\textgreater{}}, \emph{n\_bins=100}, \emph{CLs=(66.0}, \emph{95.0}, \emph{99.0)}}{}
Produce M\_max vs z plot for a given survey, cosmology and
number of z bins.

Uses the method described in Harrison \& Coles 2012 arXiv:1111.1184

n\_bins : number of redshift bins
CLs : tuple of requested confidence regions

FIXME!

\end{fulllineitems}



\section{\texttt{powspec} Module}
\label{PyCosmo:module-PyCosmo.powspec}\label{PyCosmo:powspec-module}\index{PyCosmo.powspec (module)}
(HOPEFULLY) USEFUL POWER SPECTRUM CLASS.
NOTE ALL UNITS ARE WRT h\textasciicircum{}-1

(C) IAN HARRISON
2012-
\href{mailto:IAN.HARRISON@ASTRO.CF.AC.UK}{IAN.HARRISON@ASTRO.CF.AC.UK}
\index{PowSpec (class in PyCosmo.powspec)}

\begin{fulllineitems}
\phantomsection\label{PyCosmo:PyCosmo.powspec.PowSpec}\pysiglinewithargsret{\strong{class }\code{PyCosmo.powspec.}\bfcode{PowSpec}}{\emph{cosmology}}{}~\index{choose() (PyCosmo.powspec.PowSpec method)}

\begin{fulllineitems}
\phantomsection\label{PyCosmo:PyCosmo.powspec.PowSpec.choose}\pysiglinewithargsret{\bfcode{choose}}{}{}
Choose between an Eisenstein \& HU fitting function or a CAMB power spectrum

\end{fulllineitems}

\index{display() (PyCosmo.powspec.PowSpec method)}

\begin{fulllineitems}
\phantomsection\label{PyCosmo:PyCosmo.powspec.PowSpec.display}\pysiglinewithargsret{\bfcode{display}}{}{}
Display method to show power spectrum currently working with.

\end{fulllineitems}

\index{dlnsigma\_dlnm() (PyCosmo.powspec.PowSpec method)}

\begin{fulllineitems}
\phantomsection\label{PyCosmo:PyCosmo.powspec.PowSpec.dlnsigma_dlnm}\pysiglinewithargsret{\bfcode{dlnsigma\_dlnm}}{\emph{mrange}, \emph{z}}{}
slope of root matter variance wrt log mass:
d(log(sigma)) / d(log(M))

Polynomial fit to supplied cosmology.
Returns poly1d object

\end{fulllineitems}

\index{dlnsigma\_dlnr() (PyCosmo.powspec.PowSpec method)}

\begin{fulllineitems}
\phantomsection\label{PyCosmo:PyCosmo.powspec.PowSpec.dlnsigma_dlnr}\pysiglinewithargsret{\bfcode{dlnsigma\_dlnr}}{\emph{rrange}, \emph{z}}{}
slope of root matter variance wrt log radius:
d(log(sigma)) / d(log(r))

Polynomial fit to supplied cosmology.
Returns poly1d object

\end{fulllineitems}

\index{dlnsigmadlnm\_wmap7fit() (PyCosmo.powspec.PowSpec method)}

\begin{fulllineitems}
\phantomsection\label{PyCosmo:PyCosmo.powspec.PowSpec.dlnsigmadlnm_wmap7fit}\pysiglinewithargsret{\bfcode{dlnsigmadlnm\_wmap7fit}}{\emph{lnm}}{}
Slope of root matter variance wrt log mass:
d(log(sigma)) / d(log(m))

Polynomial fit to calculation from a CAMB power spectrum 
with WMAP7 parameters

\end{fulllineitems}

\index{growth\_func() (PyCosmo.powspec.PowSpec method)}

\begin{fulllineitems}
\phantomsection\label{PyCosmo:PyCosmo.powspec.PowSpec.growth_func}\pysiglinewithargsret{\bfcode{growth\_func}}{\emph{z}}{}
initialises growth function variable
as part of PowSpec instance

\end{fulllineitems}

\index{import\_powerspectrum() (PyCosmo.powspec.PowSpec method)}

\begin{fulllineitems}
\phantomsection\label{PyCosmo:PyCosmo.powspec.PowSpec.import_powerspectrum}\pysiglinewithargsret{\bfcode{import\_powerspectrum}}{\emph{ident}, \emph{z=0.0}}{}
import power spectrum function from
a CAMB produced output file

\end{fulllineitems}

\index{interpolate() (PyCosmo.powspec.PowSpec method)}

\begin{fulllineitems}
\phantomsection\label{PyCosmo:PyCosmo.powspec.PowSpec.interpolate}\pysiglinewithargsret{\bfcode{interpolate}}{\emph{array\_1}, \emph{array\_2}}{}
returns a function that uses interpolation
to find the value of new points

\end{fulllineitems}

\index{power\_spectrum\_P() (PyCosmo.powspec.PowSpec method)}

\begin{fulllineitems}
\phantomsection\label{PyCosmo:PyCosmo.powspec.PowSpec.power_spectrum_P}\pysiglinewithargsret{\bfcode{power\_spectrum\_P}}{\emph{k}, \emph{z}}{}
returns the power spectrum P(k)

\end{fulllineitems}

\index{sigma\_fit() (PyCosmo.powspec.PowSpec method)}

\begin{fulllineitems}
\phantomsection\label{PyCosmo:PyCosmo.powspec.PowSpec.sigma_fit}\pysiglinewithargsret{\bfcode{sigma\_fit}}{\emph{rrange}, \emph{sigma\_r}}{}
\end{fulllineitems}

\index{sigma\_integral() (PyCosmo.powspec.PowSpec method)}

\begin{fulllineitems}
\phantomsection\label{PyCosmo:PyCosmo.powspec.PowSpec.sigma_integral}\pysiglinewithargsret{\bfcode{sigma\_integral}}{\emph{k}, \emph{r}, \emph{z}}{}
returns the integral required to calculate sigma
squared (Coles \& Lucchin pg.266, A.Zentner 06 eq.14)

\end{fulllineitems}

\index{sigma\_r() (PyCosmo.powspec.PowSpec method)}

\begin{fulllineitems}
\phantomsection\label{PyCosmo:PyCosmo.powspec.PowSpec.sigma_r}\pysiglinewithargsret{\bfcode{sigma\_r}}{\emph{r}, \emph{z}}{}
returns root of the matter variance, smoothed 
with a top hat window function at a radius r

\end{fulllineitems}

\index{sigma\_r\_sq() (PyCosmo.powspec.PowSpec method)}

\begin{fulllineitems}
\phantomsection\label{PyCosmo:PyCosmo.powspec.PowSpec.sigma_r_sq}\pysiglinewithargsret{\bfcode{sigma\_r\_sq}}{\emph{r}, \emph{z}}{}
integrate the function in sigma\_integral
between the limits of k : 0 to inf.

\end{fulllineitems}

\index{sigma\_r\_sq\_vec (PyCosmo.powspec.PowSpec attribute)}

\begin{fulllineitems}
\phantomsection\label{PyCosmo:PyCosmo.powspec.PowSpec.sigma_r_sq_vec}\pysigline{\bfcode{sigma\_r\_sq\_vec}\strong{ = \textless{}numpy.lib.function\_base.vectorize object at 0x34b1ad0\textgreater{}}}
\end{fulllineitems}

\index{sigma\_wmap7fit() (PyCosmo.powspec.PowSpec method)}

\begin{fulllineitems}
\phantomsection\label{PyCosmo:PyCosmo.powspec.PowSpec.sigma_wmap7fit}\pysiglinewithargsret{\bfcode{sigma\_wmap7fit}}{\emph{lnm}}{}
Root of matter variance smoothed with top hat window function on a scale
specified by log(m)

Polynomial fit to calculation from a CAMB power spectrum
with WMAP7 parameters

\end{fulllineitems}

\index{tophat\_w() (PyCosmo.powspec.PowSpec method)}

\begin{fulllineitems}
\phantomsection\label{PyCosmo:PyCosmo.powspec.PowSpec.tophat_w}\pysiglinewithargsret{\bfcode{tophat\_w}}{\emph{k}, \emph{r}}{}
Fourier transform of the real space tophat 
window function (eq.9 from A.Zentner 06)

\end{fulllineitems}

\index{transfer\_function\_EH() (PyCosmo.powspec.PowSpec method)}

\begin{fulllineitems}
\phantomsection\label{PyCosmo:PyCosmo.powspec.PowSpec.transfer_function_EH}\pysiglinewithargsret{\bfcode{transfer\_function\_EH}}{\emph{k}, \emph{z}}{}
Calculates transfer function given wavenumber

\end{fulllineitems}

\index{vd\_initialisation() (PyCosmo.powspec.PowSpec method)}

\begin{fulllineitems}
\phantomsection\label{PyCosmo:PyCosmo.powspec.PowSpec.vd_initialisation}\pysiglinewithargsret{\bfcode{vd\_initialisation}}{\emph{z}, \emph{rrange}, \emph{mrange}}{}
initialise parameters required for 
void\_distribution.py script

\end{fulllineitems}


\end{fulllineitems}



\section{\texttt{survey} Module}
\label{PyCosmo:module-PyCosmo.survey}\label{PyCosmo:survey-module}\index{PyCosmo.survey (module)}
(HOPEFULLY) USEFUL OBSERVATIONAL SURVEY CLASS.
NOTE ALL UNITS ARE WRT h\textasciicircum{}-1

(C) IAN HARRISON
2012-
\href{mailto:IAN.HARRISON@ASTRO.CF.AC.UK}{IAN.HARRISON@ASTRO.CF.AC.UK}
\index{Survey (class in PyCosmo.survey)}

\begin{fulllineitems}
\phantomsection\label{PyCosmo:PyCosmo.survey.Survey}\pysiglinewithargsret{\strong{class }\code{PyCosmo.survey.}\bfcode{Survey}}{\emph{zmin=0.0}, \emph{zmax=2.0}, \emph{lnmmin=33.845629214350737}, \emph{lnmmax=36.841361487904734}, \emph{fsky=1.0}}{}~\index{N\_in\_survey() (PyCosmo.survey.Survey method)}

\begin{fulllineitems}
\phantomsection\label{PyCosmo:PyCosmo.survey.Survey.N_in_survey}\pysiglinewithargsret{\bfcode{N\_in\_survey}}{\emph{cosm}}{}
Calculate total number of haloes expected to exist within the
observational survey window.

\end{fulllineitems}


\end{fulllineitems}



\section{\texttt{utils} Module}
\label{PyCosmo:utils-module}\label{PyCosmo:module-PyCosmo.utils}\index{PyCosmo.utils (module)}\index{as2deg() (in module PyCosmo.utils)}

\begin{fulllineitems}
\phantomsection\label{PyCosmo:PyCosmo.utils.as2deg}\pysiglinewithargsret{\code{PyCosmo.utils.}\bfcode{as2deg}}{\emph{arcsecs}}{}
Utility function for converting arcseconds to degrees

\end{fulllineitems}



\section{\texttt{void\_distribution} Module}
\label{PyCosmo:module-PyCosmo.void_distribution}\label{PyCosmo:void-distribution-module}\index{PyCosmo.void\_distribution (module)}
Python script for reproducing the
distribution of number density of voids
\index{multiplicity\_function\_jlh() (in module PyCosmo.void\_distribution)}

\begin{fulllineitems}
\phantomsection\label{PyCosmo:PyCosmo.void_distribution.multiplicity_function_jlh}\pysiglinewithargsret{\code{PyCosmo.void\_distribution.}\bfcode{multiplicity\_function\_jlh}}{\emph{sigma}, \emph{D}, \emph{void\_barrier}, \emph{collapse\_barrier}}{}
Jennings, Li \& Hu f(lnsigma) approximation

\end{fulllineitems}

\index{multiplicity\_function\_jlh\_exact() (in module PyCosmo.void\_distribution)}

\begin{fulllineitems}
\phantomsection\label{PyCosmo:PyCosmo.void_distribution.multiplicity_function_jlh_exact}\pysiglinewithargsret{\code{PyCosmo.void\_distribution.}\bfcode{multiplicity\_function\_jlh\_exact}}{\emph{sigma}, \emph{D}, \emph{void\_barrier}, \emph{collapse\_barrier}}{}
Jennings, Li \& Hu f(lnsigma) approximation

\end{fulllineitems}

\index{multiplicity\_function\_svdw() (in module PyCosmo.void\_distribution)}

\begin{fulllineitems}
\phantomsection\label{PyCosmo:PyCosmo.void_distribution.multiplicity_function_svdw}\pysiglinewithargsret{\code{PyCosmo.void\_distribution.}\bfcode{multiplicity\_function\_svdw}}{\emph{nu}, \emph{D}, \emph{void\_barrier}, \emph{collapse\_barrier}}{}
calculates equation (4) in Sheth \& van de Weygaert
approximating the infinite series in equation (1)

\end{fulllineitems}

\index{scaled\_void\_distribution() (in module PyCosmo.void\_distribution)}

\begin{fulllineitems}
\phantomsection\label{PyCosmo:PyCosmo.void_distribution.scaled_void_distribution}\pysiglinewithargsret{\code{PyCosmo.void\_distribution.}\bfcode{scaled\_void\_distribution}}{\emph{nu}, \emph{void\_barrier=-2.7}, \emph{collapse\_barrier=1.06}}{}
`A Hierarchy of Voids : Sheth \& van de Weygaert'
Reproduces a scaled distribution of void masses/sizes
shown in figure(7)

\end{fulllineitems}

\index{void\_Fr() (in module PyCosmo.void\_distribution)}

\begin{fulllineitems}
\phantomsection\label{PyCosmo:PyCosmo.void_distribution.void_Fr}\pysiglinewithargsret{\code{PyCosmo.void\_distribution.}\bfcode{void\_Fr}}{\emph{norm}, \emph{r}, \emph{ps}, \emph{max\_record=True}}{}
F(r) from Harrison \& Coles (2012)
known distribution of void radii

for max\_record=True, calculates the cumulative 
distribution \emph{upto} a given radii, otherwise
calculates EVS for small radii

\end{fulllineitems}

\index{void\_Fr\_small() (in module PyCosmo.void\_distribution)}

\begin{fulllineitems}
\phantomsection\label{PyCosmo:PyCosmo.void_distribution.void_Fr_small}\pysiglinewithargsret{\code{PyCosmo.void\_distribution.}\bfcode{void\_Fr\_small}}{\emph{norm}, \emph{r}, \emph{ps}}{}
F(r) from Harrison \& Coles (2012)
known distribution of void radii

\end{fulllineitems}

\index{void\_and\_cloud() (in module PyCosmo.void\_distribution)}

\begin{fulllineitems}
\phantomsection\label{PyCosmo:PyCosmo.void_distribution.void_and_cloud}\pysiglinewithargsret{\code{PyCosmo.void\_distribution.}\bfcode{void\_and\_cloud}}{\emph{void\_barrier}, \emph{collapse\_barrier}}{}
\end{fulllineitems}

\index{void\_fr() (in module PyCosmo.void\_distribution)}

\begin{fulllineitems}
\phantomsection\label{PyCosmo:PyCosmo.void_distribution.void_fr}\pysiglinewithargsret{\code{PyCosmo.void\_distribution.}\bfcode{void\_fr}}{\emph{norm}, \emph{r}, \emph{ps}}{}
f(r) from Harrison \& Coles (2012)
pdf of the original void distribution

\end{fulllineitems}

\index{void\_mass\_dist() (in module PyCosmo.void\_distribution)}

\begin{fulllineitems}
\phantomsection\label{PyCosmo:PyCosmo.void_distribution.void_mass_dist}\pysiglinewithargsret{\code{PyCosmo.void\_distribution.}\bfcode{void\_mass\_dist}}{\emph{m}, \emph{ps}, \emph{cosm}, \emph{z=0.0}, \emph{void\_barrier=-2.7}, \emph{collapse\_barrier=1.06}}{}
produces the differential number density of voids
wrt their characteristic mass

\end{fulllineitems}

\index{void\_norm() (in module PyCosmo.void\_distribution)}

\begin{fulllineitems}
\phantomsection\label{PyCosmo:PyCosmo.void_distribution.void_norm}\pysiglinewithargsret{\code{PyCosmo.void\_distribution.}\bfcode{void\_norm}}{\emph{ps}}{}
n\_tot from Harrison \& Coles (2012)
normalisation factor; gives the 
total comoving number density of voids

\end{fulllineitems}

\index{void\_pdf() (in module PyCosmo.void\_distribution)}

\begin{fulllineitems}
\phantomsection\label{PyCosmo:PyCosmo.void_distribution.void_pdf}\pysiglinewithargsret{\code{PyCosmo.void\_distribution.}\bfcode{void\_pdf}}{\emph{r}, \emph{norm}, \emph{ps}, \emph{V}, \emph{max\_record=True}}{}
phi(max) from Harrison \& Coles (2012)
exact extreme value pdf of the original
void distribution for a given radius

\end{fulllineitems}

\index{void\_radii\_dist() (in module PyCosmo.void\_distribution)}

\begin{fulllineitems}
\phantomsection\label{PyCosmo:PyCosmo.void_distribution.void_radii_dist}\pysiglinewithargsret{\code{PyCosmo.void\_distribution.}\bfcode{void\_radii\_dist}}{\emph{r}, \emph{ps}, \emph{z=0.0}, \emph{void\_barrier=-2.7}, \emph{collapse\_barrier=1.06}}{}
Produces the differential number density of voids 
wrt to their characteristic radius

\end{fulllineitems}



\section{Subpackages}
\label{PyCosmo:subpackages}

\chapter{Indices and tables}
\label{index:indices-and-tables}\begin{itemize}
\item {} 
\emph{genindex}

\item {} 
\emph{modindex}

\item {} 
\emph{search}

\end{itemize}


\renewcommand{\indexname}{Python Module Index}
\begin{theindex}
\def\bigletter#1{{\Large\sffamily#1}\nopagebreak\vspace{1mm}}
\bigletter{p}
\item {\texttt{PyCosmo.cluster}}, \pageref{PyCosmo:module-PyCosmo.cluster}
\item {\texttt{PyCosmo.cmb}}, \pageref{PyCosmo:module-PyCosmo.cmb}
\item {\texttt{PyCosmo.constants}}, \pageref{PyCosmo:module-PyCosmo.constants}
\item {\texttt{PyCosmo.cosmology}}, \pageref{PyCosmo:module-PyCosmo.cosmology}
\item {\texttt{PyCosmo.cosmology\_prevec}}, \pageref{PyCosmo:module-PyCosmo.cosmology_prevec}
\item {\texttt{PyCosmo.hmf}}, \pageref{PyCosmo:module-PyCosmo.hmf}
\item {\texttt{PyCosmo.hmf\_extremes}}, \pageref{PyCosmo:module-PyCosmo.hmf_extremes}
\item {\texttt{PyCosmo.powspec}}, \pageref{PyCosmo:module-PyCosmo.powspec}
\item {\texttt{PyCosmo.survey}}, \pageref{PyCosmo:module-PyCosmo.survey}
\item {\texttt{PyCosmo.utils}}, \pageref{PyCosmo:module-PyCosmo.utils}
\item {\texttt{PyCosmo.void\_distribution}}, \pageref{PyCosmo:module-PyCosmo.void_distribution}
\end{theindex}

\renewcommand{\indexname}{Index}
\printindex
\end{document}
